\makenoidxglossaries
\newglossaryentry{Sensor}
{
	name=Sensor,
	plural=Sensoren,
	description={Gerät zur Messung von physikalischen oder chemischen Eigenschaften},
}
\newglossaryentry{Cookie}
{
	name=Cookie,
	plural=Cookies,
	description={Speichert Textinformationen einer Website auf dem Computer des Nutzers},
}
\newglossaryentry{CookieNotice}
{
	name=Cookie Notice,
	description={Ein Hinweis, dass die Seite Cookies verwendet. Vorgeschrieben nach EU-Recht},
}
\newglossaryentry{Schadstoffwert}
{
	name=Schadstoffwert,
	plural=Schadstoffwerte,
	description={Schadstoffwerte geben die Anzahl von Schadstoffen in der Luft an. Dabei handelt es sich um Stoffe die schädlich für Mensch, Tiere oder Pflanzen sind},
}
\newglossaryentry{Lokalisierungsdatei}
{
	name=Lokalisierungsdatei,
	plural=Lokalisierungsdateien,
	description={Beinhalten Texte, die Platzhalter im Programm ersetzen. So können leicht alternative Übersetzungen eingesetzt werden},
}
\newglossaryentry{SmartAQNet}
{
	name=SmartAQNet,
	description={Das Smart Air Quality Network ist ein Projekt, dass den Aufbau eines intelligenten, reproduzierbaren Messnetzwerkes in der Modellregion Augsburg zum Ziel hat. Dabei werden bereits vorhandene Sensoren integriert, aber auch preiswerte neue Messtechnologien entwickelt und eingesetzt},
}
\newglossaryentry{Metadaten}
{
	name=Metadaten,
	description={Metadaten sind strukturierte Daten, die Informationen über Informationsressourcen wie zum Beispiel \glspl{Sensor} beinhalten},
}
\newglossaryentry{Kartenoverlay}
{
	name=Kartenoverlay,
	plural=Kartenoverlays,
	description={Meist farbliche Überlagerungen auf der Karte, die genutzt werden um Informationen darzustellen},
}
\newglossaryentry{Sensoroverview}
{
	name=Sensoroverview,
	plural=Sensoroverviews,
	description={Eine Übersicht über die \gls{Sensor} \gls{Metadaten} und die aktuell und in der Vergangenheit gemessenen \glspl{Schadstoffwert}},
}
\newglossaryentry{Interpolation}
{
	name=Interpolation,
	description={Verfahren zur kontinuierlichen Darstellung von diskreten Messwerten. Aus diskreten Messwerten wird dazu eine stetige Funktion abgeleitet},
}