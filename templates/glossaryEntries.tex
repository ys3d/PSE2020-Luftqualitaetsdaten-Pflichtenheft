\makenoidxglossaries
\newglossaryentry{Sensor}
{
	name=Sensor,
	plural=Sensoren,
	description={Gerät zur Messung von physikalischen oder chemischen Eigenschaften},
}
\newglossaryentry{Git}
{
	name=Git,
	description={Freie Software zur verteilten Versionsverwaltung. Git unterstützt unter anderem das speichern von \glspl{Repository} auf Servern},
}
\newglossaryentry{GitHub}
{
	name=GitHub,
	description={Großer Anbieter für online-verwaltete \gls{Git}-\glspl{Repository}},
}
\newglossaryentry{Eclipse}
{
	name=Eclipse,
	description={Eine Entwicklungsumgebung die auf die Programmiersprache Java ausgelegt ist},
}
\newglossaryentry{Smartphone}
{
	name=Smartphone,
	plural=Smartphones,
	description={Ein Mobiltelefon mit umfangreichen Computer-Funktionalitäten und Konnektivitäten}
}
\newglossaryentry{IntelliJ}
{
	name=IntelliJ IDEA,
	description={IntelliJ IDEA ist eine integrierte Entwicklungsumgebung (IDE) des Softwareunternehmens JetBrains für die Programmiersprachen Java, Kotlin, Groovy und Scala}
}
\newglossaryentry{NetBeans}
{
	name=NetBeans IDE,
	description={Netbeans ist eine vollständig in Java geschriebene Entwicklungsumgebung, die hauptsächlich auf die Entwicklung von Java ausgelegt ist},
}
\newglossaryentry{Repository}
{
	name=Repository,
	plural=Repositories,
	description={Ein verwaltetes Verzeichnis zur Speicherung und Beschreibung von digitalen Objekten. In Git bilden Projekte Repositories.}
}
\newglossaryentry{Maven}
{
	name=Maven,
	description={Ein Tool zur verwaltung von Java-Projekten. Maven wird von der Apache Software Foundation entwickelt. Durch die standardisierte Verzeichnisstruktur erlaubt Maven die Verwendung von verschiedenen Entwicklungsumgebungen für das selbe Projekt. Maven verwaltet auch Bibliotheken von Drittanbietern und vereinfacht damit das importieren von zusätzlichen Funktionalitäten},
}
\newglossaryentry{Standard-PC}
{
	name=Standard-PC,
	plural=Standard-PCs,
	description={Computer mit aktueller Hard- und Software. Aktuell heißt in diesem Fall, nicht älter als 3 Jahre}
}

\newglossaryentry{DIY}
{
	name=DIY,
	description={Eine Phrase aus dem Englischen und bedeutet übersetzt Mach es selbst. Mit der Phrase werden grundsätzlich Tätigkeiten bezeichnet, die von Amateuren ohne professionelle Hilfe ausgeführt werden.}
}
\newglossaryentry{Feinstaub}
{
	name=Feinstaub,
	description={Bezeichnet eine Teil des Staubs in der Luft, welcher sehr feine Partikelgrößen aufweist. Als Luftschadstoff wirkt sich Feinstaub negativ auf die Gesundheit aus, so kausal auf die Sterblichkeit, Herz-Kreislauf-Erkrankungen und Krebserkrankungen sowie wahrscheinlich kausal auf Atemwegserkrankungen.}
}
\newglossaryentry{Endgeraet} {
	name=Endgeraet,
	plural=Endgeraete,
    description={Das genutzte Geraet um auf die Schnittstelle zuzugreifen.},
}

\newglossaryentry{Luftqualitaetsdaten}{
    name=Luftqualitaetsdaten,
	description={Die Daten, welche von den Benutzern abgefragen werden koennen. Darunter fallen Feinstaubdaten, Luftdruck, Luftfeuchtigkeit und Temperatur.}
}
