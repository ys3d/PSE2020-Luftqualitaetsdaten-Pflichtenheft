% Produkteinsatz
\section{Produkteinsatz}
\subsection{Anwendungsbereich}
Die Schnittstelle dient der Information und Weiterbildung bezüglich \gls{Luftqualitaetsdaten}. 
Benutzer sollen in die Lage versetzt werden, sich unkompliziert
einen Überblick über die aktuellen und historischen \gls{Luftqualitaetsdaten} in Deutschland zu verschaffen.
Außerdem sollen die Benutzer ohne Vorkenntisse in Luftqualität für sie interessante Informationen aus den dargestellten
Informationen ziehen können. Daten sollen hierbei übersichtlich dargestellt und erklärt werden.
\subsection{Zielgruppe}
Die Bedienung der Schnittstelle ist möglichst einfach gestaltet, sodass jeder mit einem internetfähigen \Gls{Endgeraet} die Schnittstelle benutzen kann.
Die Zielgruppe beinhaltet Benutzer im Alter zwischen 15 und 35 Jahren, die sich über \gls{Luftqualitaetsdaten} informieren wollen. 
Die Schnittstelle soll nichtsdestotrotz von Benutzern jeden Alters genutzt werden können, die Interesse an Luftqualität haben.
Es sind keine besonderen Kenntnisse des Nutzers erforderlich.
\subsection{Betriebsbedingungen}
Folgende Bedingungen müssen erfüllt sein, damit der Benutzer auf die Schnittstelle zugreifen kann:
\begin{itemize}
    \item Der Zugang zur Datenbank, sowie zum Programm selbst steht täglich,
    24 Stunden zur Verfügung, sofern der Server, auf dem die Software installiert ist,
    dies zulässt.
    \item Der Benutzer nutzt ein betriebsbereites und internetfähiges \gls{Endgeraet} mit Farbdisplay.
    \item Das Endgerät verfügt über ein Eingabegerät.
    \item Es exisitert eine stabile Internetverbindung.
\end{itemize}
