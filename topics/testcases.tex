% Testfälle und Testszenarien
\section{Testfälle und Testszenarien}
\subsection{Testfälle}

Für jede Funktion existiert mindestens ein Testfall.
\newline
\textbf{Basis-Testfälle:}
\begin{tabularx}{\textwidth}{| c | X | c |}
	\hline
	\textbf{Nr.} & 
    \textbf{Beschreibung} &
    \textbf{Funktion}\\
    \hline
     /T010/ & Die Schnittstelle wird in einem Browser geöffnet & \\
     \hline
     /T020/ & Laden der Cookies & /D040/, /D050/, /D060/\\
     \hline
     /T030/ & Akzeptieren der Cookies &  /D040/, /D050/, /D060/\\
     \hline
     /T040/ & Laden der Deutschlandkarte beim Erstbesuch& /F010/\\
     \hline
     /T050/ & Laden einer Karte der zuletzt genutzten Umgebung des Benutzers & /D040/, /F010/\\
     \hline
     /T060/ & Anzeigen der Toolbar & /F020/\\
     \hline
     /T070/ & Aufklappen der Toolbar für Erweiterte Funktionen & /F020/\\
     \hline
     /T080/ & Anzeigen der Suchleiste & /F020/\\
     \hline
     /T090/ & Anzeigen einer Kartenlegende & /F100/\\
     \hline
     /T100/ & In der Suchleiste wird eine Stadt eingegeben. & /F030/\\
     \hline
     /T0110/ & In der Suchleiste wird eine Postleitzahl eingegeben &  /F030/\\
     \hline
     /T120/ & Auswahl eines der vier Kartenoverlays & /F050/\\
     \hline
     /T130/ & Anzeigen der Kartenoverlays & /F060/, /F070/ ,/F080/, /F090/ \\
     \hline
     /T140/ & Anzeigen der Daten von mehreren Sensoren über die Karte & /F040/\\
     \hline
     /T150/ & Markieren der Sensoren im aktuellen Kartenausschnitt & /F040/\\
     \hline 
     /T155/ & Selbstständiges markieren der Sensoren
     \hline
     /T160/ & Anzeigen der Daten eines Sensors &  /F130/, /F110/\\
     \hline
     /T170/ & Aufrufen der Histogramme & /F130/, /F140/, /F200/\\
     \hline
     /T180/ & Zoomen auf der Karte & \\
     \hline
     /T190/ & Eingabe nicht bekannter Städte/Postleitzahlen in die Suchleiste & \\
     \hline
     /T200/ & Speichern der Nutzerdaten für erneutes Laden der Schnittstelle zu einem anderen Zeitpunkt & /F160/, /F180/, /F230/ \\
     \hline
     /T210/ & Schließen der Daten der Sensoren& \\
     \hline
     /T220/ & Schließen der Schnittstelle & \\
     \hline
     /T230/ & Erneutes Laden der Schnittstelle & /F170/, /F190/, /F240/\\
     \hline
    \end{tabularx}

\textbf{Erweiterte-Testfälle:}
\begin{tabularx}{\textwidth}{| c | X | c |}
    \hline
    \textbf{Nr.} & 
    \textbf{Beschreibung} &
    \textbf{Funktion}\\
    \hline 
    /T240/ & Stresstest durch wildes drücken der Tasten & \\
    \hline      
    /T250/ & Speichern der Nutzerdaten für erneutes Laden der Schnittstelle zu einem anderen Zeitpunkt & /F160/, /F180/, /F230/ \\
    \hline  
    /T260/ & Aufrufen der DIY Anleitungen für Sensoren über die Toolbar & /F020/\\
    \hline
    /T270/ & Aufrufen des Dark-Mode über die Toolbar & /F020/\\
    \hline
    /T280/ & Änderung der Sprach-Einstellung über die Toolbar & /F020/\\
    \hline
    /T290/ & Aufrufen des Expertenmoduses & /F120/, /F210/\\
    \hline
    /T300/ & Anzeigen isopolierter Daten & \\
\end{tabularx}

\subsection{Testszenarien}

\textbf{Testszenario 1: Normaler Aufruf}
\newline
Ein unerfahrener Benutzer startet die Schnittstelle und möchte die aktuellen Feinstaubdaten seines aktuellen Standortes erfahren. Das gelingt ihm auch.
\begin{itemize} [noitemsep]
    \item 1. /T010/ Starten der Schnittstelle
    \item 2. /T040/ Laden der Deutschlandkarte
    \item 3. /T020/ Laden der Cookies
    \item 4. /T030/ Akzeptieren der Cookies
    \item 5. /T070/ Aufklappen der Toolbar
    \item 6. /T120/ Auswahl des Feinstaub-Overlays in der Toolbar
    \item 7. /T130/ Anzeigen des Overlays auf der Karte
    \item 8. /T180/ Zoomen auf der Karte
    \item 9. /T140/ Auswahl der Sensoren
    \item 10. /T160/ Daten zu diesen Sensoren werden angezeigt
    \item 11. /T210/ Schließen der Sensordaten
    \item 12. /T220/ Schließen er Schnittstelle
\end{itemize}

\textbf{Testszenario 2: Erfahrener Nutzer (Experte)}
\newline
Ein Benutzer der sich mit Sensoren etc. auskennt besucht die Schnittstelle und informiert sich über den Expertenmodus zu den Sensoren in Stadt A weiter.
\begin{itemize} [noitemsep]
    \item 1. /T230/ Erneutes Laden der Schnittstelle
    \item 2. /T050/ Laden der Karte mit dem gespeicherten Standort.
    \item 3. /T100/ Stadt A in Suchleiste eingegeben.
    \item 4. /T070/ Toolbar wird aufgeklappt
    \item 5. /T290/ Auswahl des Expertenmodus über die Toolbar
    \item 6. /T070/ Toolbar wird zugeklappt
    \item 7. /T150/ Auswahl eines Sensors.
    \item 8. /T160/ Daten zu diesem Sensor werden angezeigt
    \item 9. /T210/ Schließen der Sensordaten
    \item 10. /T220/ Schließen der Schnittstelle
\end{itemize}

\textbf{Testszenario 3: Mehrsprachig}
\newline
Ein Benutzer möchte die Schnittstelle auf Englisch angezeigt bekommen. Nach erfolgreicher Änderung wird mit Testszenario 1 weitergemacht.
\begin{itemize} [noitemsep]
    \item 1. /T230/ Erneutes starten der Schnittstelle
    \item 2. /T050/ Laden der Karte mit dem gespeicherten Standort.
    \item 3. /T070/ Toolbar wird aufgeklappt
    \item 4. /T280/ Änderung der Spracheinstellung
    \item 5. Weiter mit Testszenario 1 Schritt 6
\end{itemize}

\textbf{Testszenario 4: Eingabe einer nichtzulässigen Postleitzahl}
\newline
Der Benutzer gibt bei seiner Suche eine Postleitzahl ein, die so in Deutschland nicht existiert. Nachdem dieser darüber informiert wurde gibt er eine zulässige Postleitzahl ein.
\begin{itemize} [noitemsep]
    \item 1. /T230/ Erneutes starten der Schnittstelle
    \item 2. /T050/ Laden der Karte mit dem gespeicherten Standort.
    \item 3. /T110/ Postleitzahl in Suchleiste eingegeben.
    \item 4. /T190/ Fehlermeldung und Aufforderung zur Eingabe einer gültigen Postleitzahl
    \item 5. /T110/ Eingabe einer gültigen Postleitzahl
    \item 6. Weiter mit Testszenario 1 Schritt 6
\end{itemize}

\textbf{Testszenario 5: Historische Daten}
\newline
Ein erfahrener Benutzer möchte auf die historischen Daten ,in Form eines Histogramms, seines aktuellen Standortes zugreifen.
\begin{itemize} [noitemsep]
    \item 1. /T010/ Starten der Schnittstelle
    \item 2. /T030/ Laden der Karte mit dem gespeicherten Standort.
    \item 3. /T040/ Aktuellen Standort in Suchleiste eingegeben.
    \item 4. /T150/ Auswahl der für den Benutzer interessanten Sensoren.
    \item 5. /T170/ Im Sensordatenfenster den gewünschten Zeitraum auswählen.
    \item 6. /T220/ Schnittstelle schließen
\end{itemize}

\textbf{Testszenario 6: Dark Mode}
\newline
Ein Benutzer möchte den Dark Mode aktivieren. Nach erfolgreicher Änderung wird mit Testszenario 1 weitergemacht.
\begin{itemize} [noitemsep]
    \item 1. /T230/ Erneutes starten der Schnittstelle
    \item 2. /T200/ Laden des zuletzt genutzten Standortes.
    \item 3. /T070/ Aufklappen der Toolbar
    \item 4. /T270/ Änderung zum Dark Mode
    \item 5. Weitermachen mit Testszenario 1 Schritt 6.
\end{itemize}

\textbf{Testszenario 7: DIY Sensor}
\newline
Ein Benutzer hat Interesse an einem DIY Sensor.
\begin{itemize} [noitemsep]
    \item 1. /T230/ Erneutes Starten der Schnittstelle
    \item 2. /T050/ Laden der Karte der Umgebung des Nutzers.
    \item 3. /T070/ Aufklappen der Toolbar
    \item 4. /T260/ Öffnen der Anleitung
    \item 5. /T220/ Schließen der Schnittstelle
\end{itemize}

\textbf{Testszenario 8: Wechseln von Sensor A zu Sensor B}
\newline
Ein Benutzer will bei seinem Besuch der Schnittstelle zwischen den Unterschiedlichen Sensoren herumschalten. 
\begin{itemize} [noitemsep]
    \item 1. /T230/ Starten der Schnittstelle
    \item 2. /T050/ Laden der Karte mit dem zuletzt genutzten Standort des Nutzers.
    \item 3. /T040/ Aktueller Standort wird in die Suchleiste eingegeben
    \item 4. /T155/ Nutzer wählt manuell einen Sensor aus 
    \item 5. /T155/ Nutzer wählt mehrere Sensoren aus
    \item 6. Weiter mit Testszenario 1 Schritt 6
\end{itemize}

\textbf{Testszenario 9: Anzeigen der Daten mehrerer Sensoren}
\newline
Ein Benutzer, der die Seite bereits besucht hat, möchte die Durchschnittswerte aller ausgewählten Sensoren für seinen aktuellen Standort sehen.
\begin{itemize} [noitemsep]
    \item 1. /T230/ Starten der Schnittstelle
    \item 2. /T050/ Laden der Karte mit dem zuletzt genutzten Standort des Nutzers.
    \item 3. /T040/ Aktueller Standort wird in die Suchlesite eingegeben
    \item 4. /T150/ Sensoren werden markiert
    \item 5. /T300/ Anzeigen der isopolierten Daten
\end{itemize}
