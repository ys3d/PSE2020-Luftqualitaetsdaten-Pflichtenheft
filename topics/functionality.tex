\clearpage
% Produktfunktionen
\section{Produktfunktionen}
\subsection{Grundfunktionen}
\begin{tabularx}{\textwidth}{| c | X | c |}
\hline
        \textbf{Nr.} & 
        \textbf{Beschreibung} & 
        \textbf{Automatisch /} \\
        & & \textbf{Benutzer} \\
        \hline
        /F010/ & Anzeigen der Deutschlandkarte bei Seitenaufruf & Automatisch \\
        \hline
        /F020/ & Anzeigen der Toolbar & Automatisch \\
        \hline
        /F030/ & Suchen nach einer Stadt und automatisches markieren dieser auf der Karte (Toolbar) & Benutzer \\
        \hline
        /F040/ & Markieren der Messstationen und Sensoren im aktuellen Kartenausschnitt & Automatisch \\
        \hline
        /F050/ & Auswahl aus verschiedenen \glspl{Kartenoverlay} & Benutzer \\
        \hline
        /F060/ & Anzeigen der \glslink{Interpolation}{interpolierten} Lufttemperatur im aktuellen Kartenausschnitt als Kartenoverlay & Benutzer \\
        \hline
        /F070/ & Anzeigen der \glslink{Interpolation}{interpolierten} Luftfeuchtigkeit im aktuellen Kartenausschnitt als \gls{Kartenoverlay} & Benutzer \\
        \hline
        /F080/ & Anzeigen des \glslink{Interpolation}{interpolierten} Luftdrucks im aktuellen Kartenausschnitt als Kartenoverlay & Benutzer \\
        \hline
        /F090/ & Anzeigen der \glslink{Interpolation}{interpolierten} Feinstaubkonzentration im aktuellen Kartenausschnitt als \gls{Kartenoverlay} & Benutzer \\
        \hline
        /F100/ & Anzeigen einer Kartenlegende inklusive Farbskala des \gls{Kartenoverlay} mit gesetzlichen Grenzwerten & Benutzer \\
        \hline
        /F110/ & \gls{Sensoroverview} mit Informationen zu ausgewählten Sensoren & Benutzer \\
        \hline
        /F120/ & Anzeigen ob es sich um eine offizielle Messstation handelt (\gls{Sensoroverview}) & Benutzer \\
        \hline
        /F130/ & Anzeigen von wann der Sensorwert einer Messstation stammt (\gls{Sensoroverview}) & Benutzer \\
        \hline
        /F140/ & Zeitdiagramm zur Entwicklung der Sensorwerte einer Messstation (\gls{Sensoroverview}) & Benutzer \\
        \hline
        /F150/ & Abfragen von Sensorwerten aus einer Datenbank nach dem SensorThings Standard & Automatisch \\
        \hline
        /F160/ & Speichern des Kartenausschnitt bei Verlassen der Seite & Automatisch \\
        \hline
        /F170/ & Wiederherstellen des Kartenausschnitt beim Betreten der Seite & Automatisch \\
        \hline
        /F180/ & Speichern des ausgewählten \glspl{Kartenoverlay} bei Verlassen der Seite & Automatisch \\
        \hline
        /F190/ & Wiederherstellen des ausgewählten \glspl{Kartenoverlay} beim Betreten der Seite & Automatisch \\
        \hline
        /F200/ & Entwicklung der Schadstoffwerte in einem Zeitraum im \gls{Kartenoverlay} visualisieren / animieren & Benutzer \\
        \hline
        /F210/ & Anzeigen von allgemeinen Informationen zu Schadstoffwerten und Gesundheitsrisiken & Benutzer \\
        \hline
\end{tabularx}
\subsection{Erweiterte Funktionen}
\begin{tabularx}{\textwidth}{| c | X | c |}
\hline
        \textbf{Nr.} & 
        \textbf{Beschreibung} & 
        \textbf{Automatisch /} \\
        & & \textbf{Benutzer} \\
        \hline
        /F220/ & Auswählbarer zweiter Betriebsmodus für Experten mit detaillierten technischen Informationen zu den \glspl{Sensor} (\gls{Sensoroverview}) & Benutzer \\
        \hline
        /F230/ & Anzeigen von Informationen zur Auswirkung von Schadstoffen & Benutzer \\
        \hline
        /F240/ & Speichern des ausgewählten Betriebsmodus bei Verlassen der Seite & Automatisch \\
        \hline
        /F250/ & Wiederherstellen des ausgewählten Betriebsmodus bei Betreten der Seite & Automatisch \\
        \hline
        /F260/ & Zentrieren der Karte auf den aktuellen Standort des Nutzers & Benutzer \\
        \hline
        /F270/ & Social-Media Funktionalitäten wie das \enquote{Teilen} eines Sensorwertes auf sozialen Netzwerken & Automatisch \\
        \hline
\end{tabularx}