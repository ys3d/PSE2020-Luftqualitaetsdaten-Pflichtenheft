\section{Zielbestimmungen}
\subsection{Musskriterien}
 \begin{itemize}
	\item Der Nutzer kann auf die Webapplikation über einen Webbrowser zugreifen. 
	Der Zugriff soll sowohl über den Computer als auch über Handys möglich sein.
	\item Als erste Seite soll eine Landkarte von Deutschland geladen werden.
	Diese Karte soll Aufschluss über die Feinstaubmesswerte zu einem in der näheren Vergangenheit gemessenen Zeitpunkt geben.
	\item Der Nutzer soll auf einem Toolbar auswählen können welche Luftqualitätsdaten auf der Landkarte angezeigt werden. 
	Der Nutzer hierbei die Auswahl zwischen den Luftqualitätsdaten Luftfeuchtigkeit, Luftdruck, Temperatur und Feinstaub.
	\item Die unterschiedlichen \glspl{Luftqualitaetsdaten} sollen mit farbigen Kartenoverlay auf der Karte dargestellt werden. 
	Es gibt eine Legende für die verwendeten Farben.
	Feinstaubmesswerte sollen dabei hervorgehoben werden falls sie die gesetzlich vorgeschriebenen Grenzwerte überschreiten.
	\item Der Nutzer soll auf der Karte zoomen können. 
	Die Messwerte zur Luftqualität werden entsprechend verallgemeinert(interpoliert) bzw. detaillierter.
	\item In der detailliertesten Stufe werden die einzelnen Sensoren angezeigt. 
	Der Nutzer kann auf einen Sensor klicken und bekommt dessen Messwerte in einem Diagramm und den Sensortyp angezeigt.
	\item Der Nutzer kann auf einen Punkt auf der Landkarte klicken und bekommt dessen interpolierte Messwerte als Diagramm angezeigt.
	\item Der Nutzer kann die Postleitzahl oder den Namen eines Ortes suchen. Die Webapplikation ändert daraufhin den Landkartenausschnitt und markiert diesen auf der Karte.
	\item Auf der Webapplikation soll ebenfalls eine Zeitachse dargestellt werden. 
	Der Nutzer kann hier durch das Verschieben eines Schalters frühere Messwerte zu Luftqualität abrufen. 
	Dies soll in beliebigen Ausschnitten der Landkarte möglich sein.
	\item Der Nutzer kann ein Zeitintervall angeben. Die Webapplikation zeigt daraufhin für das gegebenen Intervall 
	und den momentanen Landkartenausschnitt, den Ort oder den \glspl{Sensor} die zeitliche Entwicklung der Luftqualität als Menge-Zeit Diagramm an.
	\item Der Nutzer kann ein Datum filtern. 
	Die Webapplikation zeigt daraufhin die entsprechenden Messwerte zur Luftqualität in dem vorliegenden Landkartenausschnitt.
	\item Der Nutzer kann sich für den vorliegenden Landkartenausschnitt den höchsten und niedrigsten Messwert anzeigen lassen.
	\item Die Webapplikation enthält eine Definition für Feinstaub.
	\item Die Webapplikation enthält eine Verlinkung zum Projekt SmartAQnet. 
\end{itemize}
\subsection{Wunschkriterien}
   \begin{itemize}
   	\item Die Webapplikation soll die Hauptgründe für Feinstaub in Deutschland in einer Graphik darstellen.
   	\item Die Webapplikation soll die Gesundheitsrisiken, die durch Feinstaub auftreten, in einer Graphik darstellen. 
	\item Der Nutzer kann einen Messwert auf der Karte auswählen. 
	Die Webapplikation zeigt daraufhin an welche Gesundheitsrisiken bei diesem spezifischen Messwert kurzfristig und langfristig auftreten.
	\item Der Nutzer kann einen Messwert auf der Karte auswählen. 
	Die Webapplikation zeigt ihm daraufhin ein alltägliches Szenario an, das eine äquivalente Feinstaubbelastung verursacht.
	\item Auf Wunsch des Nutzers soll die Webapplikation dessen Standort feststellen und die Karte entsprechend zentrieren.
	\item Durch eine Teilen-Funktion soll der Nutzer Messwerte in den sozialen Netzwerken verbreiten können.
	\item Beim Verlassen der Webapplikation soll der zuletzt betrachtete Kartenausschnitt gespeichert werden. Beim erneuten Betreten der Seite soll dieser Kartenausschnitt wiederhergestellt werden.
	\item Die Webapplikation soll eine Hilfefunktion haben, die die verschiedenen Interaktionsmöglichkeiten erklärt.
	\item Die Webapplikation enthält einen Link zu einer Bauanleitung für einen \glspl{DIY} \glspl{Sensor}.
	\item Der Nutzer kann auswählen, ob ihm die Webapplikation in deutscher oder englischer Sprache angezeigt wird.
	\item Der Nutzer soll, als zusätzliche Anzeige, die Webapplikation in einem Dark Mode oder in einem Farbenblind-Modus benutzen können.
	\item Die Webapplikation verfügt über einen Expertenmodus. In diesem werden, wenn der Nutzer auf einen Sensor klickt, typspezifische Informationen über den \glspl{Sensor} angezeigt.
	Im Expertenmodus existiert ein Filter über die Sensoren. Der Nutzer kann im Filter einstellen welcher Typ von \glspl{Sensor} auf der Landkarte angezeigt werden soll.        
\end{itemize}
\subsection{Abgrenzungskriterien}
    \begin{itemize}
	\item Bei \softwarename handelt es sich um eine reine Webapplikation.
	\item Die Webapplikation wird kein User Management betreiben.
	\item Es werden ausschließlich Daten integriert, die dem SensorThings Standard entsprechen.
	\item Es wird immer nur einer der vier möglichen Luftqualitätdaten auf der Karte angezeigt.
\end{itemize}